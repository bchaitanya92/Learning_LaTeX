% Develop a LaTeX script to create a document that contains the following table with proper labels. 

\documentclass[10pt,a4paper]{article}
\usepackage[utf8]{inputenc} % line breaks and blank spaces
\usepackage{amsmath} % fractions,bionomials,space in math mode,mathematical expressions
\usepackage{amsfonts} % Use of some math font styles
\usepackage[left=2cm,right=2cm,top=2cm,bottom=2cm]{geometry}
\usepackage{multirow} % To combine rows you need to add the multirow package
\begin{document}
    \begin{center}
   	 \begin{Large}
   		 \textbf{Table Demo}
   	 \end{Large}
    \end{center}
    \section*{Marks Table}
    \begin{tabular}{|c|c|c|c|c|c|} % This declares that six columns, separated by a vertical line, are going to be used in the table. Each c means that the contents of the column will be centred. You can also use r to align the text to the right and l for left alignment.
   	 \hline % This will insert a horizontal line on top of the table and atthe bottom too. There is no restriction on the number of times you canuse \hline.
   	 
   	 \multirow{2}{*}{S.No} & \multirow{2}{*}{USN} & \multirow{2}{*}{Student Name} & \multicolumn{3}{c|}{Marks} \\ % &amp; is a cell separator and the double-backslash \\ sets the end of this row
   	 
   	 \cline{4-6}
   	 & & & Subject1 & Subject2 & Subject3 \\
   	 \hline
   	 1 & 1KG22CS020 & Chaitanya & 88 & 77 & 97 \\
   	 \hline
   	 2 & 1KG22CS022 & Gowtham & 74 & 78 & 66 \\
   	 \hline
   	 3 & 1KG22CS028 & Charan & 88 & 82 & 79 \\
   	 \hline
    \end{tabular}
\end{document}
